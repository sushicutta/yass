\section{Pros}

\begin{itemize}

\item Docker Images sind Immutable, das garantiert, dass ein Docker Image nachdem es gebaut wurde
nicht mehr verändert wird. Das ist im Entwicklungsprozess von Vorteil, wenn man viele Engineering und
Stageing Infrastrukturen hat und die Images von einer zur nächsten transportieren muss.

\item Docker Images sind komplette Betriebsysteme. Alles was es braucht um die Applikation
zu betreiben, z.B. JRE - Java Runtime Environment, kann in einem Image mitgeliefert werden
und gehören somit ebenfalls zur Applikation.

\item Docker kann über die Kommandozeile gesteuert werden.

\item Docker fügt sich reibungslos in den Entwicklungsprozess ein, bei welchem Build-Automatisierungs
Tools eingesetzt werden. Durch die Steuerung per Kommandozeile können alle Schritte im Docker
Entwicklungsprozess mittels Shell-Scripts automatisiert werden. Solche Shell-Scripts können dann z.B.
via Jenkins im Buildprozess verwendet werden.

\item Docker Container lassen sich als \textit{root} ausführen, es kann aber auch ein spezieller
Benutzer im Dockerfile angegeben werden.

\item Der Start eines Docker Containers ist sehr schnell, meistens im Bereich < 1 Sekunde.
Gegenüber klassischer Virtuallisierung bietet das eine starke Verbesserung.

\item IntelliJ ab Version 14.1 bietet eine Docker Integration.

\item Docker Images bauen auf anderen Docker Images auf, das passt perfekt in den DRY, don't repeat
yourself, Ansatz. Es können standard Images für verschiedene Anwendungszwecke gebaut werden. So kann
für den JBoss ein Image Inhouse erstellt und gehärtet werden, welches dann von den jeweiligen
Applikationen weiterverwendet werden kann.

\item Jegliche Applikationen werden neu zu Docker Images, daraus folgt, 1 Applikation ist somit auch nur genau
1 Datei. Wir erinnern uns, man kann ein Docker Images als .tar Datei exportieren.

\item Docker bietet eine sehr gute Dokumentation aller Aspekte. Es wird ausführlich beschrieben
wie man ein Dockerfile richtig gestalten soll, oder auch wie man per Kommandozeile den Docker
Deamon richtig bedient.

\end{itemize}

\section{Cons}

\begin{itemize}

\item Yet another Dev Tool.

\item Docker kommt mit einer sehr steilen Lernkurve daher. Auch für mich war es eine Herausforderung
in dieser kurzen Zeit alles was das Doker Ökosystem bietet richtig einzuordnen.

\item Für die Ausführung eines Docker Images braucht es zwingend immer einen Docker Deamon Prozess der
in einem Linux Host Betriebsystem läuft, Das macht es für den Betrieb schwierig Docker in der
Produktion einzusetzen.

Zumindest kommt da die Frage auf, wie sicher ”bullet proof“ is der Docker Deamon Prozess? und wie
geht man mit Security Updates zur Laufzeit um, welche für den Docker Deamon Prozess erscheinen,
oder auch für die Container, welche im Rahmen eines Docker Deamon Prozesses laufen.

\item Docker existiert erst seit kurzer Zeit und es gibt noch nicht viele Entwickler und
Administratoren, welche sich professionell mit diesem Thema auseinander gesetzt haben. Somit wird
es schwer werden Leute für die Sache zu gewinnen. Vielerorts existiert eine Akzeptanzhürde bei
neuen Technologien, welche bestehendes auf den Kopf stellen könnten.

\end{itemize}