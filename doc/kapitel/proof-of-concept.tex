In einem Proof of Concept habe ich mir zum Ziel gesetzt Docker direkt auszuprobieren. Das heisst
Hands-On. Den kompletten Proof of Concept habe ich mit einem meinem Mac Book Pro - Mid 2010 durchgeführt.
Dabei habe ich nur IntelliJ IDEA 14.1.3 und das Terminal von Mac OS verwendet.

Als Proof of Concept habe ich mir ausgedacht, ich würde gerne zwei Jetty Websocket Server
mit \textit{yass} als Service Framework erstellen. Als Applikation habe ich das Tutorial
von \textit{yass} verwendet. Vor diese beiden Server soll ein Load Balancer mit
Sticky Session plaziert werden und die jeweiligen WebSocket-HTTP-Request, als Reverse Proxy
an einen der Jetty Websocket Server weiterleiten.

Da ich dabei alles Open-Source-Software verwendet habe, habe ich den erstellten Code des Proof
of Concept auf Github publiziert:

\url{https://github.com/sushicutta/yass/tree/testDocker}



