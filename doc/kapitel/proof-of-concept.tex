In einem Proof of Concept habe ich mir zum Ziel gesetzt Docker direkt auszuprobieren. Das heisst
Hands-On. Den kompletten Proof of Concept habe ich mit meinem Mac Book Pro - Mid 2010 durchgeführt.
Dabei habe ich nur IntelliJ IDEA 14.1.3 und das Terminal von Mac OS verwendet.

Als Proof of Concept habe ich mir ausgedacht, ich würde gerne zwei Jetty Websocket Server
mit \textit{yass} als Service Framework erstellen. Als Applikation habe ich das Tutorial
von \textit{yass} verwendet. Vor diese beiden Server soll ein Load Balancer mit
Sticky-Sessions plaziert werden und die jeweiligen WebSocket-HTTP-Request, als Reverse Proxy
an einen der Jetty Websocket Server weiterleiten.

Da ich dabei alles Open-Source-Software verwendet habe, habe ich den erstellten Code des Proof
of Concept auf Github publiziert:

\url{https://github.com/sushicutta/yass/tree/testDocker}

Das ganze habe ich hingekriegt und es läuft einwandfrei.
\\

\section{Fazit}

Docker ist ein sehr cooles Ökosystem. In der kurzen Zeit, habe ich es geschaft zwei Docker Images
zu erstellen, eines mit einem Java WebSocket Backend auf Basis von Jetty und yass, und eines
mit einem WebServer, der statischen Content, sprich den Client in HTML und JavaScript, ausliefert
und dann auch noch gleich das Load-Balancing mit Sticky-Sessions bietet.

Docker ist ein sehr Entwickler freundliches Tool, da man alles von der Kommandozeile her steuern
kann. Zudem ist die Dokumentation von Docker einwandfrei auf dessen Homepage verfügbar.

Ich würde mich freuen, wenn wir in Zukunft auf die Stärken von Docker zurückgreifen könnten.